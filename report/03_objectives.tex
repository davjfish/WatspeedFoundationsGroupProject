\section{Objectives}

\subsection{What are the goals of the analysis and why did you choose them?}

The goal of this analysis is to explore and better understand changes in the production of agricultural crops in Canada.
Understanding these fluctuations can yield valuable insights into the Canadian economy and the agricultural sector.

Statistics Canada collects copious amounts of data via the Census of Agriculture (REF) thus making these datasets excellent candidates for analysis in our group project.

\subsection{What question(s) do you want to answer?}

While the collections of elements affecting the total production of agricultural products are complex and multifaceted, this report we focus solely on a single variables: price.


\subsection{What hypothesis(es) do you have and what is your approach to tackle the problem?}

Our null hypothesis for this exercise is that there exist no relationship between the price of farm products and the amount of production. Our hypothesis is that there exists a relationship between the price of a commodity and its production in Canada, particulary for commodities without any Supply Management interference from the Government. 
To tackle the problem. We sourced data from the Goverment, prepared the the data, then created and validated a model of select products.  We then applied linear regression on that model to search for a relationship between price and production. 

NOTES: supply management, commodities, large investments