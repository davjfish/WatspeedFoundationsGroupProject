\section{Conclusions}

\subsection{What did you learn about your data set?}

The major takeaway from this analysis is that the production of different crops will respond to different factors, differently.
In the case of crops such as chicken meat, price appears to be a stronger driver behind crop production; when prices of meat is high, producers tend to raise more animals.
The general trend behind wheat production was similar although a more significant portion of the variation in production was unaccounted for by the current price of the commodity.
On the other hand, current hog prices clearly do not have an influence on animal production on hog farms;
This result was unexpected, especially considering the trend detected in the production of chickens.
Swine, unlike chicken meat is not under national supply management and thus more subject to fluctuation of prices due to international markets.
This fact might help account for the different responses observed in the two types of meat production.
Specifically, producers, in anticipation of highly variable prices, might decide to hedge their bets by investing in their farms regardless of the current state of the market.