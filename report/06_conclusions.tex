\section{Conclusions}

\subsection{What did you learn about your data set?}


After reviewing the liner regression results for Meat Chickens, Wheat and Hogs we found the following:

For Meat Chickens, the model intercept was found to be $26.4810$ and the coefficient price variable is $50.9459$.

\\~\\

\tabto{5cm} $y = 26.4810 + 50.9459x_1$

\begin{itemize}
    \item Where $y$ is the number of chickens, measured in millions of individuals
    \item and $x_1$ is the price of chicken meat, in dollars per kilogram
\end{itemize}

The F-statistic for this model was observed to be $12.47$; under a normal distribution the chances of observing this value are approximately $0.01\%$.
Setting our p-value to $0.05$, we would be forced to reject the null hypothesis and conclude there is indeed a relationship between the price and production of chicken meat in Canada.
The observed value for $R^2$ was $0.675$ which means that approximately 67\% of the variance of the data can be accounted for by this model.

Similarly, we found there to be a significant, positive relationship between the prices and production levels of wheat.
The model for this relationship can be described as follows:

\\~\\

\tabto{5cm} $y = 1.285e4 + 41.3883x_1$

\begin{itemize}
    \item Where $y$ is the production of wheat, in thousands of metric tonnes
    \item and $x_1$ is the price of wheat, in dollars per metric tonne
\end{itemize}

The F-statistic for this model was observed to be $8.065$; under a normal distribution the chances of observing this value are approximately $0.01\%$.
With a p-value set to $0.05$, we would be forced to reject the null hypothesis and conclude there is indeed a relationship between the price and production of wheat in Canada, albeit not as strong of a relationship as with chicken meat.
The observed value for $R^2$ was $0.287$ which means that approximately 29\% of the variance of the data can be accounted for by this model.
While this finding is significant, it is clear there are more factors involved in understanding the variations of production of wheat in Canada than just price alone.

Finally, we did not find there to be a significant relationship between the production of hogs and prices in Canada.

Based on these results, we can say that only price affects production only in some farm products produced in Canada. This combined with some understanding of of Canadian economic policy, show that affect to be stronger in products without any form of Supply Management involvement from Government. 

