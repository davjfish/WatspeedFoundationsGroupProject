\section{Data Preparation}

\subsection{What was your data source (e.g., web scraping, corporate data, a standard machine learning data set, open data, etc.)?}

The dataset used in this study is sourced from open data published by Statistics Canada, a government institution. The use of open data from a reputable government source enhances transparency and allows for reproducibility in research

\subsection{How good was the data quality?}



\subsection{What did you need to do to procure it?}

To procure the dataset, we downloaded the csv of the dataset and we read it using the pandas library.

\subsection{What tools or code did you need to use to prepare it for analysis?}

First, we segmented our analysis into three distinct categories: chicken, hogs, and wheat. The main libraries we used for our analysis were pandas, seaborn, numpy and statsmodels.api.Utilizing a dedicated helper function, we crafted time series plots for each category, with a focus on prices—a pivotal variable in our analysis. To ensure data completeness, we applied a boolean mask to handle null values. While exploring the data we recognized the historical prices of Newfoundland is very different from other provinces as you can see in Figure~\ref{fig:chicken_prices} so we decided to exclude Newfoundland from our analysis. We decided that it would be easiest to work with the data if all the prices were aggregated into a single response so we created a time series for each category where the mean prices for each category is the data and the datetime objects as the index.

\subsection{What challenges did you face?}

One notable hurdle emerged with certain features, like 'ref date', initially recorded as text, necessitating a round of feature engineering to render them immediately usable. The pricing dataset, with its varied units of measurement for a single commodity, introduced complexity, prompting us to carefully address this diversity during our analysis. The Wheat production dataset presented its own intricacies, featuring multiple entries for a given year. This intricacy made comparisons with other commodities a nuanced task. Meanwhile, the diverse data ranges of farm products posed a challenge in selecting the most pertinent ones for our analytical lens. The separation of price and production datasets added layers of complexity, urging us to integrate these disparate sources for a more comprehensive understanding. Addressing null values within the merged datasets became another crucial step in ensuring the integrity of our analysis.


\begin{figure}
    \includegraphics[width=\linewidth]{chicken_prices}
    \caption{This figure shows the raw data, and aggregated forms of chicken prices. MORE TEXT NEEDED?}
    \label{fig:chicken_prices}
\end{figure}


\begin{figure}
    \includegraphics[width=\linewidth]{chicken_production_time_series}
    \caption{This figure shows the number of chickens counted on chicken farms in Canada between 1921-2021.}
    \label{fig:chicken_production}
\end{figure}